\documentclass[12pt]{article}

\usepackage{graphicx}
\usepackage{paralist}
\usepackage{amsfonts}
\usepackage{amsmath}
\usepackage{hhline}
\usepackage{booktabs}
\usepackage{multirow}
\usepackage{multicol}
\usepackage{url}

\oddsidemargin -10mm
\evensidemargin -10mm
\textwidth 160mm
\textheight 200mm
\renewcommand\baselinestretch{1.0}

\pagestyle {plain}
\pagenumbering{arabic}

\newcounter{stepnum}

%% Comments

\usepackage{color}

\newif\ifcomments\commentstrue

\ifcomments
\newcommand{\authornote}[3]{\textcolor{#1}{[#3 ---#2]}}
\newcommand{\todo}[1]{\textcolor{red}{[TODO: #1]}}
\else
\newcommand{\authornote}[3]{}
\newcommand{\todo}[1]{}
\fi

\newcommand{\wss}[1]{\authornote{blue}{SS}{#1}}

\title{Assignment 4, Specification}
\author{SFWR ENG 2AA4}

\begin {document}

\maketitle
This Module Interface Specification (MIS) document contains modules, types and
methods for implementing the state of a game called Game of Life.


\newpage



\section* {Point Module}

\subsection* {Generic Template Module}

point(l)

\subsection* {Uses}

N/A

\subsection* {Syntax}

\subsubsection* {Exported Types}

point = ?

\subsubsection* {Exported Constants}

None

\subsubsection* {Exported Access Programs}

\begin{tabular}{| l | l | l | p{5cm} |}
\hline
\textbf{Routine name} & \textbf{In} & \textbf{Out} & \textbf{Exceptions}\\
\hline
new point & & point & none\\
\hline
new point & bool & point & none\\
\hline
state &  & bool & none\\
\hline
turn & & point & none\\
\hline

\end{tabular}

\subsection* {Semantics}

\subsubsection* {State Variables}

$live$: state of point (true or false representing alive or dead)

\subsubsection* {State Invariant}

None

\subsubsection* {Assumptions \& Design Decisions}

\begin{itemize}
\item The point(live) constructor is called for each object instance before any
  other access routine is called for that object.  
\item The point() constructor create a dead point to make the coding for other 
  modules easier to write. Though it violates the essential property of the point
  object, since this could be achieved by calling point(false), this
  method is provided as a convenience to write the code. 

\end{itemize}

\subsubsection* {Access Routine Semantics}

new point():
\begin{itemize}
\item transition: $live := false$

\item output: $\mathit{out} := \mathit{self}$
\item exception: none
\end{itemize}

\noindent new point($l$):
\begin{itemize}
\item transition: $live := l$

\item output: $\mathit{out} := \mathit{self}$
\item exception: none
\end{itemize}



\noindent state():
\begin{itemize}
\item output: $out := live$ 
\item exception: none
\end{itemize}



\noindent turn():
\begin{itemize}
\item transition: $live := \lnot live$
\item output: $out$ := $self$
\item exception: none
\end{itemize}


\newpage

\section* {map ADT Module}

\subsection*{Template Module}

map

\subsection* {Uses}

\noindent point

\subsection* {Syntax}

\subsubsection* {Exported Access Programs}

\begin{tabular}{| l | l | l | l |}
\hline
\textbf{Routine name} & \textbf{In} & \textbf{Out} & \textbf{Exceptions}\\
\hline
new map  &  & map &\\
\hline
new map  & seq of (seq of point) & map & bad size\\
\hline
new map & name of the file in string  & map & invalid argument, bad size\\
\hline
generate &  & map & \\
\hline
tab &$\mathbb{N}$, $\mathbb{N}$  &  map & out of range \\
\hline
output  & & file & invalid\_argument\\
\hline

\end{tabular}

\subsection* {Semantics}

\subsubsection* {State Variables}

$board$: seq of (seq of point) \textit{\# game board}\\

\subsubsection* {State Invariant}

$min\_size = 25$\\
$max\_size = 50$\\

\subsubsection* {Assumptions \& Design Decisions}

\begin{itemize}

\item The map constructor is called before any other access
  routine is called on that instance. 

\item The points on the edge of the map dies in next generation because of
	the void that consumes them

\item The map(string fn) function reads file that named fn, the file only contains "0" and "*" representing $true$ and $false$ respectively.
	The function throws $bad\_size$, which is a custom exception.

\item The $output()$ function is a void type function. However, it does output a file named "output.txt", which is the text based 
	graph of the gameboard in this generation. 

\item The map can be a square or a rectangle. As long as the length of sides are in beween (min\_size , max\_size)

\item There are three functions that can initialize a new map. Though violating the preperty of being 
	essential, it provides different ways get a new board of game. The map(vector(vector(point)) and map(filename) are neccesary because
	those are the only two way to create a new board with inputs. The map() function is just for users to get a empty board in the easiest way. 



\end{itemize}

\subsubsection* {Access Routine Semantics}

\noindent map():
\begin{itemize}
\item transition: 
$board := s \text{ such that } (|s| = max\_size) \land (\forall\, i
\in [0...max\_size-1] : s[i] =  t \text{ such that } (|t| = max\_size \land (\forall\, j
\in [0...max\_size-1] : t[j] = \text{point(false)}$
\item output: $out$ := $self$
\item exception: None
\end{itemize}


\noindent map($\mathit{seq\_of(seq\_of(point) ) \,  b}$):
\begin{itemize}
\item transition: 
$board := b$
\item output: $out$ := $self$
\item exception: $exc := (|b| < min\_size) \lor (|b| > max\_size)\lor (|b[0]| < min\_size) \lor (|b[0]| > max\_size) \lor
	(\exists i \in [0... |b[0]|-2] : |b[i]| \ne |b[0]| ) \Rightarrow \text{bad\_size}$
\end{itemize}


\noindent map($string \, fn$):
\begin{itemize}
\item transition: 
$board := s \text{ such that } (|s| = \text{ num of rows in } fn ) \land \\ (\forall\, i
\in [0... \text{ (num of rows in) } fn-1] : s[i] =  t \text{ such that } (|t| = \text{ (num of columns in) } fn \land \\ (\forall\, j
\in [0... \text{ (num of columns in( } fn-1] : ("*" \Rightarrow t[j] = \text{point(false)}  \lor "0" \Rightarrow t[j] = \text{point(true)}  )$
\item exception: $exc := |b| < min\_size \lor |b| > max\_size\lor |b[0]| < min\_size \lor |b[0]| > max_size \lor
	(\exists i \in [0... |b[0]|-2] : |b[i]| \ne |b[0]| ) \Rightarrow \text{bad\_size}$
\end{itemize}


\noindent generation():
\begin{itemize}
\item transition: 
$x,y \in \mathbb{N} | \forall x \in [0...|board|-1] \land \forall y \in [0...|board[0]|-1] $ \\
\begin{tabular}{|p{3cm}|l|}
\hhline{|-|-|}
$ board[x][y] = point(false)$ & $x=0 \lor y=0 \lor neighbor( board,x , y) <= 2 \lor  neighbor( board,x , y) > 3$\\
\hhline{|-|-|}
$ board[x][y] = point(true)$ & $board[x][y].neighbor = 3$\\
\hhline{|-|-|}
\end{tabular}
\item output: $out$ := $self$
\item exception: None
\end{itemize}

\noindent tab(x, y):
\begin{itemize}
\item transaction: $ board[x][y] = board[x][y].turn()$
\item output:  $out$ := $self$
\item exception: $(x < 1 \lor y < 1 \lor x > |board.size| - 1 \lor y > |board[0].size| - 1) \Rightarrow out\_of\_range$
\end{itemize}

\noindent output():
\begin{itemize}
\item output: a file named "output.txt" such that: \\ $ (\forall\, x
\in [0... |board|-1] \land \forall \, y \in [0... |board[0]|-1] : \\ (board[x][y].state() = false \Rightarrow \text{write "*" to file}) \lor \\
 (board[x][y].state() = true \Rightarrow \text{write "0" to file}))$

\item exception: None

\end{itemize}



\subsection*{Local Types}

None

\subsection*{Local Functions}

\noindent  neighbor( board,x , y) : $seq\_of (seq\_of(point)) \times \mathbb{N} \times \mathbb{N} \rightarrow  \mathbb{N}$ 
\begin{itemize}

\item output:$ + (i,j \in int | \exists i \in [-1, 0, 1] \land \exists j \in [-1, 0 ,1] : board[x+i][y+j].state() = true )$

\end{itemize}


\newpage

\section*{Critique of Design}

The $point(live)$ module is not neccesary for the project, since we can just construct map with $vector<vector<bool>>$. However, I think it
is neccesary to have a function $turn()$, that can easily change the dead point to alive and the alive point to dead. The $point$ medule also 
gives client a better clarification how the game works than just using $bool$. \\



\end {document}